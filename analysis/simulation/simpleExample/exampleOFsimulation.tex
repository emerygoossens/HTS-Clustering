% Created 2016-02-07 Sun 21:07
\documentclass[11pt]{article}
\usepackage[utf8]{inputenc}
\usepackage[T1]{fontenc}
\usepackage{fixltx2e}
\usepackage{graphicx}
\usepackage{longtable}
\usepackage{float}
\usepackage{wrapfig}
\usepackage{rotating}
\usepackage[normalem]{ulem}
\usepackage{amsmath}
\usepackage{textcomp}
\usepackage{marvosym}
\usepackage{wasysym}
\usepackage{amssymb}
\usepackage{hyperref}
\tolerance=1000
\usepackage{listings}
\usepackage[usenames,dvipsnames]{color}
\usepackage{underscore}
\usepackage{parskip}
\usepackage{lmodern}
\usepackage{parskip}
\setlength{\parindent}{0pt}
\usepackage{underscore}
\textwidth 16cm
\oddsidemargin 0.5cm
\evensidemargin 0.5cm
\author{Heejung Shim}
\date{\today}
\title{"How to simulate data with three clusters from real data"}
\hypersetup{
  pdfkeywords={},
  pdfsubject={},
  pdfcreator={Emacs 23.1.1 (Org mode 8.2.10)}}
\begin{document}

\maketitle
\tableofcontents


\section{Introduction}
\label{sec-1}
Ideas for simulating data from real data has been described in Chapter 5 in Supplementary Material of \href{http://projecteuclid.org/euclid.aoas/1437397106}{Shim and Stephens 2015}. We will use DNase-seq data presented in \href{http://projecteuclid.org/euclid.aoas/1437397106}{Shim and Stephens 2015}.
You can obtain data from \href{https://github.com/heejungshim/WaveQTL}{WaveQTL repo}. Simulation scheme uses functions provided in \href{https://github.com/heejungshim/utils_shim/}{utils\_shim repo}.

\section{Simulation scheme}
\label{sec-2}
\lstset{breaklines=true,showspaces=false,showtabs=false,tabsize=2,basicstyle=\ttfamily,frame=single,keywordstyle=\color{Blue},stringstyle=\color{BrickRed},commentstyle=\color{ForestGreen},columns=fullflexible,language=R,label= ,caption= ,numbers=none}
\begin{lstlisting}
## set up working directory (you need to change this).
setwd('/home/hjshim/d/projects/HTS-Clustering');

## read path to repositories 
WaveQTL.repodir =scan(".WaveQTL.repodir.txt",what=character())
utils_shim.repodir =scan(".utils_shim.repodir.txt",what=character())

## read functions for simulation. We will use a function `sample.from.Binomial.with.Overdispersion'. See the file for detaied description of arguments.
source(paste0(utils_shim.repodir, "/R/utils_multiscale.R"))

## read DNase-seq data. This DNase-seq data has been obtained as average of two strands. For simulation, we convert them to count data. 
pheno.dat = as.matrix(read.table(paste0(WaveQTL.repodir, "/data/dsQTL/chr17.10160989.10162012.pheno.dat")))
pheno.dat = ceiling(pheno.dat)

## read genotype data and convert them to three genotypes
sel_geno_IX = 11
geno.dat = as.numeric(read.table(paste0(WaveQTL.repodir, "/data/dsQTL/chr17.10160989.10162012.2kb.cis.geno"), as.is = TRUE)[sel_geno_IX, -(1:3)])
geno.dat = round(geno.dat)
table(geno.dat)
## geno.dat
##  0  1  2 
## 28 30 12 

## create three DNase-seq data sets for three genotypes.
DNase0 = pheno.dat[geno.dat==0,]
DNase1 = pheno.dat[geno.dat==1,]
DNase2 = pheno.dat[geno.dat==2,]

## sample 10 curves from each cluster. I put over.dispersion = 0.001, but you could chnage this value. 
num.sam = 10
DNase = DNase0
total.count = as.numeric(apply(DNase, 2, sum))
mu.sig = rep(1/dim(DNase)[1], dim(DNase)[2])
over.dispersion=0.001
data0 = sample.from.Binomial.with.Overdispersion(num.sam, total.count, mu.sig, over.dispersion)

num.sam = 10
DNase = DNase1
total.count = as.numeric(apply(DNase, 2, sum))
mu.sig = rep(1/dim(DNase)[1], dim(DNase)[2])
over.dispersion=0.001
data1 = sample.from.Binomial.with.Overdispersion(num.sam, total.count, mu.sig, over.dispersion)

num.sam = 10
DNase = DNase2
total.count = as.numeric(apply(DNase, 2, sum))
mu.sig = rep(1/dim(DNase)[1], dim(DNase)[2])
over.dispersion=0.001
data2 = sample.from.Binomial.with.Overdispersion(num.sam, total.count, mu.sig, over.dispersion)
\end{lstlisting}
% Emacs 23.1.1 (Org mode 8.2.10)
\end{document}
